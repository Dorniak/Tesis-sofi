Casi de casualidad, buscando temática para mi Trabajo Fin de Carrera me topé con la investigación. No sabía hasta qué punto me iba a enganchar hasta que un día logré que un modelo matemático sobre un sistema de suspensión funcionase en Matlab. Desde entonces han pasado 12 años en busca de la innovación por la mejora de la sociedad, en concreto en el campo de la automoción. Han sido muchos, y muy diversos los proyectos que me han acogido, al igual que muchas las personas que han estado presentes en mi carrera investigadora, porque la tesis doctoral es algo que empieza mucho antes de tu primer día de doctorado. 

A día de hoy me siento agradecida de todas las puertas cerradas, ya que gracias a ello terminé en uno de los lugares top de mi lista de trabajos soñados desde que terminé la carrera en la UMH de Elche, el Instituto de Investigación del Automóvil Francisco Aparicio Izquierdo (INSIA). El INSIA es un lugar mágico que atrae a personas fantásticas, todas y cada una de ellas con sus taras, y sin las cuales esta tesis no tendría sentido. Gracias a todos por vuestro granito de arena.

El gracias más grande se lo debo a mi Director de tesis, Felipe Jiménez, por todas las oportunidades, la confianza, el acompañamiento y lo más difícil, la inspiración en los momentos oscuros. Gracias a Eugenio y a Elisa por la formación complementaria y el crecimiento en áreas tan interdisciplinares como las comunicaciones y la psicología. 

A mi tutor de tesis emocional e inhibidor de pensamientos, Manolo, este trabajo no sería el mismo sin ti, y probablemente yo no seguiría de una sola pieza. A mis padres, mis hermanos y mi cuñada por la confianza, por enseñarme que la base es el trabajo y la constancia. A mi abuela, a los Sánchez y a los Mateo. Gracias a mis xunguis de Alicante, siempre dispuestos a mantearme entre halagos y amor. 

A mis niños del LABIE, sujetos experimentales y creadores de ensayos locos, gracias por dejarme hacer lo que me diera la gana con vosotros. Especial gracias a Óscar y a Samu, por el cariño, el conocimiento y por la amistad, vivan los arduinos, la cerveza y los LEDs. A Alber por esta última etapa intensa, a Miguel, Edgardo, José siempre malo, Alberto Cruz, Víctor, Luid, a mi energética Choni, y a todos los que habéis contribuido de alguna forma en este proyecto y que me dejo en el tintero.

A Nacho por acogerme en su familia y hacer que tuviera la mejor estancia de mi vida en Suecia, en el Statens väg-och transportforskningsinstitut (VTi). Son innumerables los recuerdos bonitos y lo mucho que he podido aprender de psicología, de los suecos y de la investigación. Gracias a la ayuda del Programa Propio de la Universidad Politécnica de Madrid para poder realizar una estancia digna.

A mis referentes investigadores con, sin y en proceso de tesis, a los Marañosianos, a los escapistas, a mis hermanas del aire, y a todos aquellos que con cariño te hacen la maldita y angustiosa pregunta: ¿cómo llevas la tesis? Gracias, pero no escucharla más será el mayor de los descansos.

Y, por último, gracias a mi nueva familia en la ETSIDI, por la acogida y por acompañarme en el descubrimiento de lo que puede ser la siguiente pasión, aún en dudas, la enseñanza.



% Aquí va la cita célebre si la hubiese. Si no, comentar la(s) linea(s) siguientes
% \chapter*{}
% \thispagestyle{empty}
% \setlength{\leftmargin}{0.5\textwidth}
% \setlength{\parsep}{0cm}
% \addtolength{\topsep}{0.5cm}
% \begin{flushright}
% \small\em{
% Si consigo ver más lejos\\
% es porque he conseguido auparme\\ 
% a hombros de gigantes
% }
% \end{flushright}
% \begin{flushright}
% \small{
% Isaac Newton.
% }
% \end{flushright}
% \cleardoublepage %salta a nueva página impar
