\section{Conclusiones}
En esta Tesis se ha buscado generar conocimiento sobre la caracterización del comportamiento del conductor para optimizar el proceso de toma de decisiones en los algoritmos aplicados a los vehículos autónomos. Las conclusiones del trabajo realizado se pueden estructurar en tres bloques, correspondiendo a los capítulos y objetivos secundarios en los que se divide la investigación.

\underline{Comportamiento visual en conducción}

El comportamiento del conductor ha sido estudiado en situaciones complejas en vías de alta capacidad, que implican un cambio de carril, como son las incorporaciones y los adelantamientos, a través del estudio de su comportamiento visual. Las incorporaciones a vías de alta capacidad, como autovías o autopistas, son situaciones altamente exigentes para los conductores, donde la experiencia y la edad influyen decisivamente en la carga mental. La variación pupilar respecto a la tasa base se estimó en un 5\%, a pesar de que a nivel global dicha situación no se percibe como destacablemente estresante, según más de la mitad de la población encuestada.  

En línea con esta conclusión se desarrolló de una aplicación de ayuda a la incorporación, cuya ubicación fue determinada mediante la obtención de mapas de calor de las miradas y el análisis de las variables oculares durante la maniobra. Del total de las fijaciones realizadas se obtuvo que un 30\% se realizaron en los espejos retrovisores, destacando paralelamente un punto caliente situado en la esquina interior superior del mismo. En coherencia con los demás resultados relacionados con la duración y el número de fijaciones, se concluye que el conductor adquiere suficiente información desde esta zona gracias a la visión periférica, evitando desviar la atención del frente lo máximo posible. El diseño de la interfaz fue evaluado mediante métricas de aceptación del usuario y variables atencionales del conductor, encontrando relaciones directas entre ambos métodos y reforzando la importancia del diseño de los sistemas de asistencia al conductor para su aceptación por parte del conductor.  

En relación con la gestión atencional en adelantamientos, se estudió la influencia en la realización de una tarea secundaria en conducción manual y parcialmente automatizada, obteniendo que los conductores no la consideraron como distractora, posiblemente debido a la transferencia de conocimiento derivada del uso de dispositivos móviles y pantallas de navegación. Previo a la maniobra de adelantamiento, se analizó el periodo de anticipación a la maniobra, determinando que este intervalo se puede estimar en 10 segundos, en los cuales el área correspondiente al carril central y la tarea fueron las más observadas, en conducción manual y en parcialmente automatizada respectivamente. Estas áreas también destacaron en el estudio de la secuencia visual, seguidas de espejo retrovisor izquierdo y el panel de mandos, confirmando el sesgo de fijación central para la zona del carril por el que circula el conductor. Comparativamente la secuencia entre frente y tarea fueron superiores en conducción manual, indiferentemente de su orden, y el área frente con espejo izquierdo y panel de mandos, lo fueron en conducción parcialmente automatizada, señalando claramente que la activación del sistema permite al conductor derivare su atención a otras áreas.   

En el análisis de las miradas a la tarea por fases en el adelantamiento no se encontraron efectos significativos, pero sí se observaron diferencias entre los modos en la duración media de las miradas a la tarea, la cual fue ligeramente superior en modo autónomo, y el número de las mismas, donde se observa el efecto opuesto. Este hecho indica que, en autónomo, el conductor observa la tarea distendidamente, donde sus miradas son más largas pero menos habituales, y en manual manifiesta un comportamiento contrario. La demanda cognitiva destacó en la zona de carril izquierdo en conducción manual, señalando la complejidad de la situación, ya que el conductor confía en que el retorno al carril derecho no supone mucha complejidad e intenta derivar su atención a la realización de la tarea. 

\underline{Fusión sensorial} 

El sistema de seguimiento visual ha sido complementado con un sistema de seguimiento de la cabeza con objeto de poder referenciar la mirada del conductor en el sistema de coordenadas global. La integración de los sistemas de percepción del conductor con los sistemas de percepción del entorno ha sido evaluada mediante un análisis de correlación, cuyos resultados han sido buenos a nivel individual pero mejorables en la detección conjunta, debido a las condiciones específicas de los ensayos y el método de clusterización empleado.  

Durante los ensayos la mayor parte de las miradas del conductor ocurrieron dentro del ángulo central de visión, el cual depende de la velocidad en cada instante, sin embargo, solo un tercio de las miradas totales se realizaban al obstáculo en concreto. Este hecho respalda la idea de que el conductor puede obtener información de la escena sin necesidad de una visión detallada, lo que enfatiza la importancia de la visión periférica. Esta característica también plantea desafíos a la hora de recopilar datos precisos sobre la información visual que los conductores realmente registran y utilizan para tomar decisiones en cada momento.  

\underline{Modelo de toma de decisiones}

Las variables que se evalúan en el proceso de la toma de decisiones en conducción son diferentes en función del perfil estudiado. Variables como la aceleración propia mínima, máxima y la velocidad media del carril izquierdo tuvieron correlación con perfiles de conductores más jóvenes y menos experimentados, mostrando una conducción más dinámica y nerviosa. Por otro lado, los conductores más experimentados primaron variables relacionadas con la seguridad, evitando maniobras arriesgadas incluso si implica una reducción en su velocidad. En este grupo destaca una conciencia sobre las propias capacidades de los vehículos y la dinámica de vehículos de gran tamaño, como camiones o autobuses.  

Para la validación del modelo de toma de decisiones se obtuvieron datos de conducción naturalista a través de seis tipologías de cambio de carril, en función de la configuración de vehículos situados en el carril izquierdo. La aceptación del hueco difirió entre las maniobras propuestas, resaltando que, en el cambio de carril entre dos vehículos, los conductores ajustan su velocidad con el vehículo delantero y dejan un promedio de 4 veces más espacio con el vehículo trasero, indicando una asimetría en la distancia de seguridad mantenida con ambos vehículos. 

El comportamiento visual mostró que la anticipación de la maniobra se caracterizó por un aumento de miradas al espejo retrovisor entre los segundos 10-20 previo al cruce de la línea central, acorde con los resultados expuestos en el primer bloque. La densidad media de miradas en el periodo de anticipación se estimó entre 3 y 6 durante para este intervalo temporal.  

La anticipación a la ejecución de la maniobra de cambio de carril se modelizó a través de la caracterización de tres instantes temporales, resumidos en tiempo máximo de aceleración libre, tiempo mínimo de inicio de la intención de cambio y tiempo de seguridad previo a la ejecución. Analizando dichos instantes en cada maniobra planteada se advirtió que el tiempo de ejecución más bajo se produjo en la maniobra C1, cambio de carril sin vehículos, hecho muy coherente siendo esta maniobra la más sencilla debido a la inexistencia de vehículos en el carril izquierdo. En los resultados obtenidos para las maniobras C2, cambio de carril por delante del primero, C5, cambio de carril entre el segundo y el tercer vehículo, y C6, cambio de carril el último de la formación, se observó que los conductores no necesitaron tanto tiempo para analizar el entorno como en las demás maniobras, apuntando a un factor de impaciencia por el cambio de carril, en el caso concreto de C2, o a una ayuda por parte de los vehículos adyacentes para realizar la maniobra, en el caso de C5 y C6. La intención de cambio de carril se identificó mediante el estudio de relaciones de velocidad entre los vehículos, y se consideró que la principal razón para el cambio era la oportunidad de aumentar la velocidad propia y equipararla con la de los vehículos que circulan por el carril izquierdo. 

La validación del modelo con los ensayos realizados en conducción real tuvo una tasa de éxito del 82.22\%. El margen de error podría residir en una confianza excesiva del modelo frente al conductor, observándose en la ejecución de algunas maniobras como C6, cambio de carril el último de la formación, la más discordante, donde el modelo consideró en varias ocasiones que la maniobra C4, cambio de carril entre el primer y el segundo vehículo, era más adecuada para esa situación.

\section{Futuros desarrollos}
Considerando las conclusiones obtenidas en esta Tesis Doctoral, se identifican diversas áreas de interés para futuras investigaciones en el campo de la comprensión del comportamiento del conductor para la mejora de la toma de decisiones en conducción autónoma.  

En relación a la fusión del sistema de percepción del entorno con el de percepción de la mirada del conductor, se reconoce la necesidad de mejorar su precisión, considerando la implementación de más cámaras que aporten redundancia y robustez al sistema. Futuras investigaciones podrían centrarse en el desarrollo de dispositivos más avanzados y precisos, que permitan una mejor adquisición de datos y una mayor fiabilidad en la interpretación de la información recopilada. 

Además de la captura del movimiento ocular, es importante explorar otras técnicas que permitan completar la comprensión de la información que el conductor procesa en cada momento. Un ejemplo podría ser el uso del método \enquote{think aloud}, donde se pide al conductor que verbalice sus pensamientos y decisiones durante la conducción. Esta técnica podría proporcionar información adicional sobre el proceso de toma de decisiones y enriquecer así la comprensión de las acciones del conductor, siendo una solución de bajo coste y de aplicación sencilla con otros procedimientos existentes. 

Por otro lado, la aplicación de técnicas de inteligencia artificial en el modelado del comportamiento del conductor puede ser una línea de investigación interesante, dando lugar a modelos más complejos y precisos que tengan en cuenta una variedad de factores y características individuales del conductor, y que puedan adaptarse a diferentes escenarios de conducción. 

Si bien este estudio se ha centrado en maniobras que implican un cambio de carril, como las incorporaciones y los adelantamientos en vías de alta capacidad, sería destacable ampliar la investigación a otras maniobras y escenarios de conducción más complejos, como la conducción por vías urbanas. Elementos como las intersecciones y las rotondas, así como las situaciones de tráfico intenso y la presencia de peatones, plantean desafíos adicionales para el conductor, por lo que su estudio ayudaría comprender el proceso de decisión en este entorno para el desarrollo de estrategias de conducción más seguras y eficientes en tráfico urbanos. 

La principal consideración metodológica en futuros desarrollos es la realización de ensayos experimentales con una muestra de conductores más representativa, fortaleciendo la validez y la confiabilidad de los resultados obtenidos. Trabajar con una muestra de conductores lo más amplia posible permitirá obtener conclusiones más robustas sobre las relaciones entre variables en el proceso de la toma de decisiones. De igual forma, la utilización de herramientas estadísticas más adecuadas y sofisticadas en la investigación del comportamiento del conductor puede abrir nuevas perspectivas y aportar una mayor profundidad y precisión en el análisis de los datos.  

En futuros desarrollos relacionados con el factor humano en conducción autónoma, puede ser de gran relevancia el estudio la implementación de reglas de conducción en un entorno de realidad virtual o simulador de conducción. Esta estrategia permitiría evaluar la eficacia de los desarrollos y los algoritmos propuestos en un entorno controlado y seguro, reduciendo los recursos utilizados y la implicación humana. En este sentido, la implementación de algoritmos de conducción autónoma en un entorno de simulación se considera el paso previo a su implementación en un vehículo autónomo real en un entorno cerrado. Este enfoque permite realizar pruebas exhaustivas y detalladas, donde se pueden evaluar y perfeccionar las reglas y algoritmos de conducción sin poner en riesgo la seguridad de conductores y peatones de los entornos reales de tráfico. 

\section{Difusión de resultados}\label{ch7}
Las siguientes publicaciones recogen resultados parciales de la Tesis Doctoral, comprendiendo publicaciones en revistas de alto índice de impacto, así como la participación en congresos de índole nacional e internacional. Los avances realizados a lo largo de la investigación han sido ampliamente difundidos y reconocidos en la comunidad científica, contribuyendo a una mayor comprensión y aplicación de los resultados obtenidos.

\subsection{Revistas indexadas en JCR}
\begin{itemize}
    \item Jiménez, F., Naranjo, J. E., Sanchez-Mateo, S., Serradilla, F., Pérez, E., Hernández, M., Ruiz, T. (2018). Communications and Driver Monitoring Aids for Fostering SAE Level-4 Road Vehicles Automation. Electronics, 7(10), 228. ISSN 2079-9292. Q3 
    \item Sanchez–Mateo, S., Pérez–Moreno, E., Jiménez, F. (2020). Driver Monitoring for a Driver-Centered Design and Assessment of a Merging Assistance System Based on V2V Communications. Sensors, 20(5582). ISSN 1424-8220. Q1 
    \end{itemize}
    
\subsection{Congresos}

\textbf{\underline{2018}}
\begin{itemize}
    \item Sanchez-Mateo, S., Perez-Moreno, E., Jiménez, F., Naranjo, J. E., Perez Flores, C. E., Antoñazas Teruel, J. (2018). Study of a driver assistance interface for merging situations on highways.2018 IEEE International Conference on Vehicular Electronics and Safety (ICVES). Madrid, España. 
    \item Sanchez Mateo, S., Clavijo, M., Diaz- Alvarez, A., Jiménez, F. (2018). Interface design for an assistance system focused on high attentional load situations. 25th ITS World Congress. Copenhague, Dinamarca. 
    \end{itemize}

\textbf{\underline{2019}}
\begin{itemize}
    \item Sanchez-Mateo, S., Perez-Moreno, E. Jiménez, F., Serradilla, F., Cruz-Ruiz, A., de la Fuente Tamayo, S. (2019). Validation of an assistance system for merging maneuvers in highways in real driving conditions. 16th European Automotive Congress (EAEC). Minsk, Bielorrusia. 
    \end{itemize}

\textbf{\underline{2021}}
\begin{itemize}
    \item Sanchez-Mateo, S., De la Puente, G., Martín, A., De la Fuente, S., Jiménez, F. (2021). Cognitive load influence of driving HMIs. Industriales Research Meeting, Madrid, España. 
    \item De la Fuente Tamayo, S., Sanchez-Mateo, S., Jiménez, F. (2021). Identificación automática de la percepción de elementos en la carretera por parte del conductor mediante sistemas de visión y láser. XXIII Congreso Nacional de Ingeniería Mecánica. Jaén, España. 
    \item Jiménez, F., Astudillo, A., Monsalve, B., Sanchez-Mateo, S., Sesmero, M. P., Armingol, J. M., Fernández Andrés, J., Naranjo, J. E., Sanchis, A., Aliane, N. (2021). Distributed decision support system for cooperative connected and autonomous driving in complex environments. 27th ITS World Congress. Hamburgo, Alemania.
\end{itemize}

\textbf{\underline{2022}}
\begin{itemize}
    \item Sanchez-Mateo, S., Jiménez, F. (2022). Variables atencionales aplicadas a la toma de decisiones en conducción autónoma. IV Campus Científico del Foro de Ingeniería del Transporte. Madrid, España.
    \item Sanchez-Mateo, S., Valle-Barrio, A., Díaz-Álvarez, A., Jiménez, F. (2022). Evaluación de modelo determinista para conducción autónoma a través del comportamiento visual. XV Congreso Iberoamericano de Ingeniería Mecánica.Madrid, España.
\end{itemize}

\textbf{\underline{2023}}
\begin{itemize}
    \item Cruz-Ruiz, A., Jiménez, F., Naranjo, J. E., Sanchez-Mateo, S. (2023). Modular open source automated platform for autonomous driving bus. 11th Young Researchers Seminar, 44, Lisboa, Portugal.  
    \item Sanchez-Mateo, S., Cruz-Ruiz, A., Jiménez, F. (2023). Estudio experimental del hueco aceptable para el perfeccionamiento de la maniobra de cambio de carril en vehículos autónomos. XV Congreso de Ingeniería de Transporte. La Laguna, Tenerife, España. 
\end{itemize}

\newpage
\section*{Conclusions and future works }

\subsection*{Conclusions}
In this Thesis the research has aimed to generate knowledge on the characterization of driver behavior in order to optimize the decision-making process in the algorithms applied to autonomous vehicles. The conclusions of the work carried out can be structured in three blocks, corresponding to the chapters and secondary objectives in which the research is divided. 

\underline{Driving visual behavior }

Driver behavior has been studied in complex situations on high-capacity roads, involving lane changes, such as merging and overtaking, through the study of their visual behavior. High-capacity road merging, such as highways or freeways, are highly demanding situations for drivers, where experience and age have a decisive influence on the mental load. The pupillary variation with respect to the prime rate was estimated at 5\%, despite the fact that on an overall level such a situation is not perceived as noticeably stressful, according to more than half of the surveyed population.  

In line with this conclusion, an assistant system for merging was developed, whose location was determined by obtaining heat maps of the gazes and analyzing the ocular variables during the maneuver. Of the total number of fixations made, it was found that 30\% were made on the rear-view mirrors, highlighting a particular hot spot located in the upper inner corner of the mirror. Consistent with the other results related to the duration and number of fixations, it is concluded that the driver acquires sufficient information from this area thanks to the peripheral vision, avoiding diverting attention from the front as much as possible. The interface design was evaluated by means of user acceptance metrics and driver attentional variables, finding direct relationships between both methods and reinforcing the importance of the design of driver assistance systems for driver acceptance.  

In relation to attentional management in overtaking, the influence on the performance of a secondary task in manual and partially automated driving was studied, obtaining that drivers did not consider it as distracting, possibly due to the transfer of knowledge derived from the use of mobile devices and navigation screens. Prior to the overtaking maneuver, the period of anticipation to the maneuver was analyzed, determining that this interval can be estimated at 10 seconds, in which the area corresponding to the center lane and the task were the most observed, in manual and partially automated driving respectively. These areas also stood out in the visual sequence study, followed by the left rearview mirror and the dashboard, confirming the central fixation bias for the area of the lane in which the driver is driving. Comparatively, the sequence between front and task were superior in manual driving, regardless of their order, and the front area with left mirror and dashboard were superior in partially automated driving, clearly indicating that the activation of the system allows the driver to focus his attention to other areas.   

In phase analysis of task glances during overtaking, no significant effects were found, but differences were observed between modes in the average duration of glances at the task, which was slightly higher in autonomous mode, and the number of glances, where the opposite effect was observed. This fact indicates that, in autonomous mode, the driver observes the task at ease, where his glances are longer but less frequent, and in manual mode he shows the opposite behavior. Cognitive demand emphasized in the left lane zone in manual driving, indicating the complexity of the situation, as the driver trusts that the return to the right lane does not involve much difficulty and tries to direct his attention to the performance of the task. 

\underline{Sensor fusion }

The visual tracking system has been complemented with a head tracking system in order to be able to reference the driver's gaze in the global coordinate system. The integration of the driver perception systems with the environment perception systems was evaluated by means of a correlation analysis, whose results have been positive at the individual level but could be improved in the combined detection, due to the specific conditions of the tests and the clustering method used.  

During the tests, most of the driver's glances occurred within the central viewing angle, which depends on the speed at each instant, however, only one third of the total glances were made at the specific obstacle. This fact supports the idea that the driver can obtain scene information without detailed vision, which emphasizes the importance of peripheral vision. This feature also presents challenges in collecting accurate data on the visual information that drivers actually register and use to make decisions at any given moment. 

\underline{Decision-making model }

The variables evaluated in the decision-making process in driving are different according to the profile studied. Variables such as minimum and maximum own acceleration and average left lane speed were correlated with younger and less experienced driver profiles, showing more dynamic and nervous driving. On the other hand, more experienced drivers prioritized variables related to safety, avoiding risky maneuvers even if it implies a reduction in their speed. In this group, an awareness of the vehicles' own capabilities and the dynamics of large vehicles, such as trucks or buses.  

For the validation of the decision-making model, naturalistic driving data were obtained through six lane change typologies, depending on the configuration of vehicles located in the left lane. Gap acceptance differed between the proposed maneuvers, highlighting that, in lane-changing between two vehicles, drivers adjust their speed with the front vehicle and leave an average of 4 times more space with the rear vehicle, indicating an asymmetry in the safety distance maintained with both vehicles. 

The visual behavior showed that the anticipation of the maneuver was characterized by an increase in glances to the rearview mirror between seconds 10-20 prior to crossing the center line, consistent with the results presented in the first block. The average density of glances in the anticipation period was estimated to be between 3 and 6 during this time interval. 

The anticipation to the lane change maneuver execution was modeled through the characterization of three-time instants, summarized in maximum free acceleration time, minimum time of change intention initiation and safety time prior to the execution. Analyzing these instants in each maneuver, it was noticed that the lowest execution time occurred in maneuver C1, lane change without vehicles, a very coherent fact, since this maneuver is the simplest due to the absence of vehicles in the left lane. In the results obtained for maneuvers C2, lane change ahead of the first vehicle, C5, lane change between the second and third vehicle,  and C6, lane change last in the formation, it was observed that drivers did not need as much time to analyze the environment as in the other maneuvers, pointing to an impatience factor for the lane change, in the specific case of C2, or to an allowance by surrounding vehicles to perform the maneuver, in the case of C5 and C6. The intention to change lanes was identified by studying the speed ratios between vehicles, and the main reason for the change was considered to be the opportunity to increase one's own speed to match that of vehicles in the left lane. 

The validation of the model with the tests performed in real driving had a success rate of 82.22\%. The margin of error could reside in an excessive confidence of the model in front of the driver, observed in the execution of some maneuvers such as C6, lane change the last of the formation, the most discordant, where the model considered on several occasions that the maneuver C4, lane change between the first and second vehicle, was more appropriate for that situation.

\subsection*{Future works }

Considering the conclusions obtained in this Doctoral Thesis, several areas of interest for future research in the field of understanding driver behavior for the improvement of decision-making in autonomous driving are identified.  

In relation to the fusion of the environment perception system with the driver's gaze perception system, the need to improve its accuracy is recognized, considering the implementation of more cameras that provide redundancy and robustness to the system. Future research could focus on the development of more advanced and accurate devices, allowing better data acquisition and greater reliability in the interpretation of the information collected. 

In addition to eye movement capture, it is important to explore other techniques to complete the understanding of the information that the driver is processing at any given moment. An example could be the use of the \enquote{think aloud} method, where the driver is asked to verbalize his thoughts and decisions while driving. This technique could provide additional information about the decision-making process and thus enrich the understanding of the driver's actions, being a low-cost solution and simple to apply with other existing procedures. 

On the other hand, the application of artificial intelligence techniques in modeling driver behavior may be an interesting line of research, leading to more complex and accurate models that take into account a variety of individual driver factors and characteristics, and that can be adapted to different driving scenarios. 

Although this study has focused on maneuvers involving lane changes, such as merging and overtaking on high-capacity roads, it would be noteworthy to extend the research to other more complex maneuvers and driving scenarios, such as driving on urban roads. Elements such as intersections and roundabouts, as well as heavy traffic situations and the presence of pedestrians, pose additional challenges for the driver, so their study would help to understand the decision process in this environment for the development of safer and more efficient driving strategies in urban traffic. 

The main methodological consideration in future developments is to conduct experimental test with a more representative sample of drivers, strengthening the validity and reliability of the results obtained. Working with as large a sample of drivers as possible will allow more robust conclusions to be drawn about the relationships between variables in the decision-making process. Similarly, the use of more appropriate and sophisticated statistical tools in the investigation of driver behavior can open up new perspectives and provide greater depth and precision in data analysis.  

In future developments related to the human factor in autonomous driving, it may be of great relevance to study the implementation of driving rules in a virtual reality environment or driving simulator. This strategy would allow evaluating the effectiveness of the proposed developments and algorithms in a controlled and safe environment, reducing the resources used and human involvement. In this sense, the implementation of autonomous driving algorithms in a simulation environment is considered the step prior to their implementation in a real autonomous vehicle in a closed environment. This approach allows thorough and detailed testing, where driving rules and algorithms can be evaluated and refined without compromising the safety of drivers and pedestrians in real traffic environments.