El factor humano en conducción presenta numerosos desafíos relativos a la seguridad, los cuales podrían ser abordados eficientemente a través de tecnologías relacionadas con la conducción autónoma. En los últimos años ha habido un avance significativo en los sistemas de asistencia al conductor, proporcionando funciones de apoyo y mejorando la seguridad y la comodidad en carretera. En relación con la automatización total, los fabricantes de automóviles han emprendido una carrera tecnológica en busca del vehículo sin conductor invirtiendo recursos significativos en investigación y desarrollo. Sin embargo, existen ciertas barreras que ralentizan la integración de estos vehículos en el parque automovilístico actual. 

Uno de los factores más determinantes es la aceptación social, condicionada directamente por la confiabilidad de estos sistemas. A pesar de las múltiples pruebas en entornos cerrados y el aumento de sensores, es difícil abarcar el total de la casuística de accidentes de tráfico que se pueden producir en tráfico real. Muchos de los problemas detectados en el ámbito de la conducción autónoma se relacionan con problemas que un conductor humano podría resolver con relativa sencillez, apuntando a una falta de reglas en el sistema de decisión. En este aspecto, los estudios naturalistas desempeñan un papel fundamental en el desarrollo de algoritmos de toma de decisiones basados en el comportamiento humano, ya que los vehículos carecen de cierta información que los conductores adquieren de forma natural.

Es por ello que el estudio del comportamiento del conductor es crucial para el desarrollo de sistemas que interactúen con vehículos de conducción manual. Comprender los procesos cognitivos seguidos por un conductor y su estado en diversos entornos perfeccionará el diseño de las reglas de decisión ante diferentes maniobras, optimizando la toma de decisiones en conducción autónoma.

El objetivo principal de la tesis es mejorar la caracterización del comportamiento del conductor mediante el análisis de la percepción visual en maniobras complejas realizadas en vías de alta capacidad, como son autovías o autopistas. A lo largo de este estudio, se evalúa la influencia de las variables atencionales del conductor ante diferentes niveles de asistencia a la conducción, observando la repetición de ciertos patrones visuales en función del entorno. 

La integración de la información visual del conductor en un modelo de toma de decisiones naturalista permitió una validación exitosa del mismo con ensayos experimentales realizados en tráfico real. Previamente, se realizó una fusión sensorial del sistema de percepción del entorno con el sistema de seguimiento visual, permitiendo la proyección automática de la mirada del conductor en el entorno exterior. Los desarrollos realizados generaron adicionalmente conocimiento destacable en relación con la anticipación de la maniobra de cambio de carril y el hueco aceptable para el desarrollo de modelos de conducción. 

Las conclusiones de la Tesis doctoral contribuyen a una mejora de la modelización del comportamiento del conductor y aportarán un enfoque más naturalista al desarrollo de algoritmos de toma de decisiones, con el objetivo de mejorar la integración de la conducción autónoma en el tráfico mixto.


\vspace{10pt}
\textbf{Palabras clave:} Conducción autónoma; Toma de decisiones; Comportamiento humano; Seguimiento ocular; Modelo de conducción.
