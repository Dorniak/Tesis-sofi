Human factor in driving presents numerous safety-related challenges, which could be efficiently addressed through technologies related to autonomous driving. In recent years, there has been significant progress in driver assistance systems providing support functions and improving safety and comfort on the road. In relation to full automation, automakers have embarked on a technological race in pursuit of the driverless vehicle by investing significant resources in research and development. However, there are certain barriers that slow down the integration of these vehicles into the current automotive fleet.

One of the most determining factors is social acceptance, directly conditioned by the reliability of these systems. Despite multiple tests in closed environments and the increase in sensors, it is difficult to encompass the full range of traffic accident scenarios that can occur in real-world traffic. Many of the problems identified in the field of autonomous driving are related to issues that a human driver could resolve relatively easily, pointing to a lack of rules in the decision-making system. In this regard, naturalistic studies play a fundamental role in the development of decision-making algorithms based on human behavior, as vehicles lack certain information that drivers naturally acquire.

That is why the study of driver behavior is crucial for the development of systems that interact with manually driven vehicles. Understanding the cognitive processes followed by a driver and their state in various environments will enhance the design of decision-making rules for different maneuvers, optimizing decision-making in autonomous driving.

The main objective of the thesis is to improve the characterization of driver behavior through the analysis of visual perception in complex maneuvers performed on high-capacity roads, such as highways or expressways. Throughout this study, the influence of driver attentional variables is evaluated under different levels of automation, observing the repetition of certain visual patterns depending on the environment.

The integration of driver visual information into a naturalistic decision-making model allowed for successful validation through experimental tests conducted in real traffic. Prior to that, a sensor fusion was performed between the environment perception system and the visual tracking system, enabling the automatic projection of the driver's gaze onto the external environment. The developments also generated notable knowledge regarding the anticipation of lane-changing maneuvers and the acceptable gap for the development of driving models.

The findings of the PhD Thesis contribute to an improvement in the modeling of driver behavior and will provide a more naturalistic approach to the development of decision-making algorithms, with the aim of enhancing the integration of autonomous driving in mixed traffic.

\vspace{10pt}
\textbf{Keywords:} Autonomous driving; Decision-making; Human behavior; Eye tracking; Driving model.
